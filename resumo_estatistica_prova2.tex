\documentclass[10pt]{article}
\usepackage[utf8]{inputenc}
\usepackage[T1]{fontenc}
\usepackage[brazilian]{babel}
\usepackage{comment} 
\usepackage{geometry}
\geometry{
    a4paper,
    total={170mm,257mm},
    left=20mm,
    top=20mm,
 }
\setlength{\parindent}{4em}
\setlength{\parskip}{1em}
\renewcommand{\baselinestretch}{1.2}

\title{Resumo Prova 2 Estatística}
\author{Guilherme Christopher Michaelsen Cardoso}
\date{\today}

\begin{document}
\maketitle
\abstract{Probabilidade. Variáveis aleatórias discretas. Variáveis aleatórias contínuas.  Distribuições
    amostrais e estimação de parâmetros.}
\section{Probabilidade}
\par Iniciamos o estudo dos processos que envolvem variabilidade, aleatoriedade ou incerteza, 
construindo modelos matemáticos para facilitar a análise. Esses modelos normalmente 
são construídos à partir de suposições sobre o processo, mas também podem se basear
em dados observados no passado.

\par Pessoas costumam tomar decisões em função dos fatos que têm maior probabilidade de ocorrer, como
por exemplo:

\begin{itemize}
    \item Se o céu está nublado, então há chance considerável de chover.
    \item Se um inspetor de qualidade está observando as peças produzidas por uma máquina, e 
        verifica que elas estão saindo fora do padrão, então é provável que a máquina continue
        produzindo peças fora do padrão.
    \item Se em uma determinada família há recorrência de problemas cardíacos, existe chance
        maior de pessoas daquela família serem afetadas, portanto, exames preventivos devem
        ser feitos com frequência.
\end{itemize}

\par Outro aspecto é a incerteza inerente às decisões que podem ser tomadas sobre determinado problema.
Nos exemplos anteriores, mesmo que o céu esteja nublado, existe a chance de não chover. A máquina pode
estar funcionando bem, e as peças fora do padrão podem ter sido produzidas por eventos casuais, etc.

\par A possibilidade de quantificar a incerteza associada a cada fato permite tomar decisões mais
eficientes. A teoria do cálculo de probabilidades permite obter uma quantificação da incerteza associada
a um ou mais fatos. 

\par Os modelos \textit{probabilísticos} são aplicados em situações que envolvem algum tipo de
\textit{incerteza} ou \textit{variabilidade}, por exemplo, na presença de algum experimento aleatório.

\par São exemplos de experimentos aleatórios:
\begin{itemize}
    \item Lançamento de um dado honesto (não viciado) e observação da face voltada para cima.
    \item Observação dos diâmetros, em mm, dos eixos produzidos em uma metalúrgica
    \item Número de mensagens transmitidas corretamente por dia em uma rede de computadores.
\end{itemize}

\par Nos casos em que os possíveis resultados de um experimento aleatório podem ser listados,
um \textbf{modelo probabilístico} pode ser entendido como a listagem desses resultados, acompanhados
de suas respectivas probabilidades.

\subsection{Espaço amostral e eventos}

\begin{itemize}
    \item O conjunto de \textit{todos} os possíveis resultados do experimento é chamado de espaço 
    amostral e é denotado pela letra grega \begin{math} \Omega \end{math}.
\end{itemize}

\textbf{Exemplos de experimentos aleatórios e seus espaços amostrais:}
\begin{enumerate}
    \item Lançamento de um dado e observação da face voltada para cima:
        \begin{math} \Omega = \{1,2,3,4,5,6\} \end{math}
    \item Sacada de carta num baralho comum (52 cartas) e observação do naipe:
        \begin{math} \Omega = \{copas, espadas, ouros, paus\} \end{math}
    \item Número de mensagens transmitidas corretamente por dia em uma rede de computadores:
        \begin{math} \Omega = \{0,1,2,3,...\} \end{math}
    \item Observação do diâmetro, em mm, de um eixo produzido em uma metalúrgica:
        \begin{math} \Omega = \{d, \end{math} tal que \begin{math} d > 0\} \end{math} 
\end{enumerate}

\par O espaço amostral pode ser:
\begin{itemize}
    \item Finito, formado por um número limitado de resultados possíveis (casos 1 e 2)
    \item Infinito enumerável, formado por um número infinito de resultados os quais
        podem ser listados (caso 3)
    \item Infinito, formado por intervalos de números reais (caso 4).
\end{itemize}

\par Um espaço amostral é dito \textbf{discreto} quando for finito ou infinito enumerável
e contínuo quando é formado por intervalos de números reais.

\par Chamamos de \textbf{evento} qualquer subconjunto do espaço amostral.
\begin{math} A \end{math} é um evento \begin{math} \iff A \subseteq \Omega \end{math}

\par Exemplo: seja o esperimento do lançamento de um dado, com 
    \begin{math}\Omega = \{1,2,3,4,5,6\}\end{math}.\\ São exemplos de eventos:
\begin{itemize}
    \item A = número par do dado = \{2,4,6\}
    \item B = número maior que 2 do dado = \{3,4,5,6\}
    \item C = número 6 = \{6\}
\end{itemize}
    \par Como um evento é um subconjunto do espaço amostral, então todos os conceitos
    de teoria de conjuntos podem ser aplicados a eventos. (união, intersecção, complementar).
    \par Dois eventos são ditos \textbf{mutualmente exclusivos} se, e somente se, eles não
    puderem ocorrer, isto é, se \begin{math}A \cap B = \emptyset \end{math}.
\subsection{Definições de probabilidade}
\subsubsection{Definição clássica de probabilidade}
\par Se um experimento aleatório tem \begin{math}n\end{math} resultados \textit{igualmente}
    prováveis e \begin{math}n_A\end{math} desses resultados pertencem a certo evento A, então
    a probabilidade de ocorrência do evento A será:
    \begin{displaymath}
        P(A) = \frac{n_A}{n}
    \end{displaymath}
\subsubsection{Definição experimental de probabilidade}
\par Seja um experimento aleatório com espaço amostral \begin{math}\Omega\end{math} e um 
    evento A de interesse. Suponha que o experimento seja repetido n vezes e o evento A
    ocorreu n(a) vezes. A frequência relativa do evento A é dada por:
    \begin{displaymath}
        f(a) = \frac{n(A)}{n}
    \end{displaymath}
    À medida que o experimento é repetido mais vezes, sob as mesmas condições,
    a frequência relativa do evento A tenderá a ficar cada vez mais próxima
    da probabilidade de ocorrência do evento A. Mais especificamente:
    \begin{displaymath}
        P(A) = \lim_{n \to \infty} f(a) = \lim_{n \to \infty} \frac{n(A)}{n}
    \end{displaymath}
\end{document}
