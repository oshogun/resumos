\documentclass[10pt]{article}
\usepackage[utf8]{inputenc}
\usepackage[T1]{fontenc}
\usepackage[brazilian]{babel}
\usepackage{comment} 
\usepackage{geometry}
\usepackage{listings}
\geometry{
    a4paper,
    total={170mm,257mm},
    left=20mm,
    top=20mm,
 }
\setlength{\parindent}{4em}
\setlength{\parskip}{1em}
\renewcommand{\baselinestretch}{1.2}

\title{Resumo Prova 3 SO}
\author{Guilherme Christopher Michaelsen Cardoso}
\date{\today}

\begin{document}
\lstset{language=C}
\maketitle

\section{Sistemas de Arquivos}
\begin{itemize}
    \item Programas de computador precisam armazenar e recuperar informação.
    \begin{itemize}
        \item Quando um processo está rodando, ele pode armazenar uma quantidade
        limitada de informação dentro do seu próprio espaço de endereçamento.
        \item Porém, a capacidade desse armazenamento está limitada ao tamanho
        do espaço de endereçamento virtual. 
        \item Para alguns programas, esse tamanho é adequado; para outros, é muito
        pequeno.
        \item Outro problema de se utilizar o espaço de enderaçmento virtual é
        que ao encerrar o processo, a informação é perdida. Para muitas aplicações,
        a informação deve ser retida por semanas, meses, ou mesmo para sempre. 
        Não se aceita que essa informação desapareça quando um processo termina ou
        sofre um \textit{crash}.
        \item O terceiro problema é que frequentimente é necessário que múltiplos
        processos acessem partes da informação ao mesmo tempo. Por isso, a informação
        deve ser independente dos processos.
    \end{itemize}
    \item Assim, surgem 3 requisitos essenciais para armazenamento de informação à 
    longo prazo:
    \begin{enumerate}
        \item Deve ser possivel armazenar uma quantidade muito grande de informação
        \item A informação deve sobreviver ao término do processo que a está utilizando
        \item Múltiplos processos devem ser capazes de acessar a informação de uma vez.
    \end{enumerate}
    \item Dispositivos comuns de armazenamento não-volátil incluem:
    \begin{enumerate}
        \item Discos magnéticos
        \item SSDs (não possuem partes móveis que podem quebrar, oferecem acesso rápido)
        \item Fitas e discos ópticos (usados tipicamente para backup devido a sua baixa
        performance).
    \end{enumerate}
    \item Pode-se pensar no disco como sendo uma sequência linear de blocos de tamanho
    fixo, capazes de suportar duas operações:
    \begin{enumerate}
        \item Ler o bloco \textit{k}
        \item Escrever no bloco \textit{k}
    \end{enumerate}
    \begin{itemize}
        \item Na verdade existem mais operações, mas essas duas poderiam, em princípio,
        resolver o problema do armazenamento à longo prazo.
        \item Na realidade, essas são operações muito inconvenientes, especialmente
        em sistemas grandes usado por muitas aplicações e, possivelmente, por muitos
        usuários (por exemplo, em um servidor). Alguns problemas que surgem são:
        \begin{enumerate}
            \item Como encontrar informação?
            \item Como garantir que um usuário não leia os dados de outro?
            \item Como saber quais blocos estão livres?
        \end{enumerate}
        e existem muito mais.
    \end{itemize}
    \item Da mesma forma que o SO abstraiu o conceito do processador para criar o 
    conceito do processo, e abstraiu o conceito da memória física para oferecer
    a processos espaços de endereçamento virtuais, os problemas relacionados a disco
    podem ser resolvidos com uma nova abstração: \textbf{o arquivo}. 
    \begin{itemize}
        \item Esses três conceitos (processos, espaços de endereçamento e arquivos)
        são os mais importantes em Sistemas Operacionais.
    \end{itemize}
    \item \textbf{Arquivos} são unidades lógicas de informação criadas pelos processos.
    Um disco normalmente contém milhares ou até mesmo milhões deles, cada um 
    independente dos outros. Assim como os espaços de endereçamento modelam a RAM,
    arquivos modelam o disco.
    \item Processos podem ler arquivos existentes e criar novos arquivos de acordo com 
    a necessidade. Informação armazenada em arquivos deve ser persistente, isto é, não
    pode ser afetatda pela criação e término dos processos. Um arquivo deve desaparecer
    apenas quando seu usuário explicitamente o remove. Apesar de operações de leitura
    e escrita de arquivos serem as mais comuns, existem muitas outras.
    \item Arquivos são gerenciados pelo sistema operacional. A forma em que eles são
    estruturados, nomeados, acessados, usados, protegidos, implementados e gerenciados
    são tópicos importantes em design de sistemas operacionais. A parte do SO que lida
    com arquivos é chamada de \textbf{sistema de arquivos}.
\end{itemize}
\subsection{Arquivos}
\abstract{Nessa subseção, observa-se os arquivos do ponto de vista do usuário,
isto é, como eles serão usados e quais propriedades eles terão.}
\subsubsection{Nomeação de Arquivos}
\begin{itemize}
    \item Um arquivo é um mecanismo de abstração, que provê uma forma de armazenar
    informação no disco e recuperá-la depois. 
    \begin{itemize}
        \item Isso deve ocorrer de forma a esconder do usuário os detalhes sobre como
        e onde essa informação é armazenada, e como os discos funcionam.
    \end{itemize}
    \item Quando um processo cria um arquivo, ele dá a esse arquivo um nome.
    \begin{itemize}
        \item Quando o processo termina, esse arquivo continua a existir e pode ser acessado
        por outros processos usando seu nome.
    \end{itemize}
    \item As regras para nomear arquivos variam de sistema para sistema.
    \begin{itemize}
        \item Todos os SOs atuais permitem strings de 1 a 8 letras como nome de arquivo.
        \item Frequentemente digitos numéricos e caracteres especiais também são permitidos.
        \item Muitos sistemas de arquivos suportam nomes longos de até 255 caracteres.
        \item Alguns sistemas de arquivos realizam distinção entre letras maiúsculas
        e minúsculas (UNIX, por exemplo), e outras não (MS-DOS, por exemplo).
        \begin{itemize}
            \item Um adendo sobre sistemas de arquivos: Tanto o Windows 95 quanto o 
            Windows 98 usam o sistema de arquivos do DOS, chamado de \textbf{FAT-16},
            herdando muitas de suas propriedades (como a construção de nomes de arquivos).
            Windows 98 introduziu algumas extensões ao FAT-16, criando o \textbf{FAT-32},
            mas esses dois são similares. As versões posteriores do Windows suportam os 
            sistemas de arquivos FAT, que já são obsoletos, porém, esses sistemas já 
            utilizam um sistema mais avançado (\textbf{NTFS}), que possui propriedades
            diferentes (como nomes de arquivo em Unicode). No Windows 8 ainda existe um
            segundo sistema de arquivos, chamado \textbf{ReFS}, mas esse é usado apenas
            na versão para servidores. Nesse capítulo, ao mencionar aos sistemas de arquivos
            do MS-DOS ou FAT, está se referenciando o uso dos sistemas FAT-16 e FAT-32 no 
            Windows. Ainda existe um outro sistema baseado em FAT, conhecido como \textbf{exFAT},
            que é uma extensão do FAT-32 otimizada para pendrives e sistemas de arquivos grandes.
        \end{itemize}
        \item Muitos sistemas operacionais suportam nomes de arquivos dividodos em duas partes,
        separadas por um ponto (ex.: "prog.c"). A parte depois do ponto é chamada de 
        \textbf{extensão de arquivo} e geralmente indica algo sobre o arquivo. No MS-DOS, por
        exemplo, nomes de arquivos contém de 1 a 8 caracters, mais uma extensão adicional de
        1 a 3 caracteres. No UNIX, o tamanho da extensão é definido pelo usuário, e um arquivo
        pode ter até mesmo mais de uma extensão (exemplo homepage.html.zip).
        \begin{itemize}
            \item No caso do UNIX e alguns outros sistemas, as extensões de arquivo são apenas
            convenções e não são obrigatórias pelo sistema operacional. Um arquivo chamado
            file.txt pode ser algum tipo de arquivo de texto, mas o nome serve mais
            para informar o usuário do que para prover qualquer informação para o computador.
            Por outro lado, um compilador C pode exigir que os arquivos de entrada estejam 
            com extensão .c, mas o SO não se importa.
            \item Essas convenções são especialmente úteis quando o mesmo programa pode 
            lidar com diversos tipos de arquivos.
            \item Por outro lado, o Windows utiliza as extensões para identificar quais programas
            são "donos" de cada extensão. Quando o usuário chama um arquivo, o SO invoca o programa
            associado à sua extensão.
        \end{itemize}
    \end{itemize}
\end{itemize}
\subsubsection{Estrutura de Arquivos}
Arquivos podem ser estruturados de várias formas. Três possibilidades são comuns:
\begin{enumerate}
    \item Uma sequência não-estruturada de bytes. O SO não se preocupa com o que está no arquivo.
    Tudo que ele vê são os bytes. Qualquer significado deve ser imposto pelos programas à nível de
    usuário. Tanto o UNIX quanto o Windows usam essa aproximação. 

    Fazer com que o SO considere os arquivos como nada mais do que sequências de bytes provê a 
    maior flexibiliade. Programas de usuário podem colocar qualquer coisa em seus arquivos e nomeá-los
    de qualquer forma que achar conveniente. O SO não ajuda, mas também não atrapalha. Para usuários
    que querem fazer coisas incomuns, isso é muito importante. Todas as versões do UNIX (incluindo
    Linux e OSX) e o Windows usam esse modelo de arquivos.

    \item Uma sequência de records de tamanho fixo, cada um com uma estrutura interna. I idéia central
    de um arquivo ser uma sequencia de records é a idéia de que a operação de leitura retorna um record
    e a operação de escrita substitui ou acrescenta um record. Essa técnica era usada quando cartões perfurados
    de 80 colunas eram comuns, e os SOs usavam arquivos que consistiam de records de 80 caracteres.

    \item Uma árvore de records, não necessariamente todos do mesmo tamanho, cada um contendo um campo 
    \textbf{chave} em uma posição fixa do record. Essa árvore é ordenada no campo chave, para permitir
    a busca rápida de uma determinada chave. Esse tipo de arquivo é usado em computadores mainframe
    para processamento de dados comerciais.
\end{enumerate}

\subsubsection{Tipos de Arquivos} 

\begin{itemize}
    \item Vários SOs suportam vários tipos de arquivos.
    \item UNIX e Windows, por exemplo, possuem arquivos regulares e diretórios. O UNIX também possui 
    arquivos especiais de caracteres e blocos. 
    \begin{itemize}
        \item \textbf{Arquivos regulares} contém informação de usuário. 
        \item \textbf{Diretórios} são arquivos de sistema que mantém a estrutura do sistema de arquivos.
        \item \textbf{Arquivos especiais de caracteres} são relacionados à entrada/saída e são usados
        para modelar dispositivos seriais de I/O como terminais, impressoras e redes.
        \item \textbf{Arquivos especiais de bloco} são usados para modelar discos.
    \end{itemize}
    \item Arquivos regulares geralmente são ASCII ou binários. Arquivos ASCII contém linhas de texto.
    Em alguns sistemas, cada linha é terminada por um caractere de carriage return. Em outros, o caractere
    de line feed é usado. Em alguns sistemas (e.g., Windows), ambos são usados. Linhas não precisam ser todas
    do mesmo comprimento.
    \item A grande vantagem de arquivos ASCII é que eles podem ser mostrados e impressos como são, e podem ser
    editados com qualquer editor de texto. Além disso, se grandes números de programas usam arquvios ASCII 
    para entrada e saída, é facil conectar a saída de um programa na entrada de outro (como nos pipelines
    de shell).
    \item Arquivos binários são aqueles que não são ASCII. Tentar imprimir seu conteúdo na tela geraria
    um monte de lixo aleatório. Geralmente eles tem alguma estrutura interna conhecida apenas pelos programas
    que os utilizam. 
    \begin{itemize}
        \item Um exemplo de arquivo binário seria um executável do UNIX. Apesar de tecnicamente ser apenas
        uma sequência de bytes, o SO vai executar um arquivo apenas se el tiver um formato apropriado, com
        cinco sessões: header, text, data, relocation bits e symbol tabe. O header começa com um 
        \textbf{magic number} que identifica o arquivo como um executável (para impedir a execução acidental
        de um arquivo que não esteja nesse formato), seguido pelos tamanhos das várias partes do arquivo, o
        endereço no qual a execução começa, e alguns bits de flag. Em seguida estão os segmentos de texto e dados
        do programa em si. Eles são carregados em memória e relocados usando os bits de relocação. A tabela de símbolos
        é usada apenas para debug.
        \item Outro exemplo seria um arquivo compactado, também do UNIX. Ele consiste de uma coleção de rotinas de 
        bibliotecas compiladas mas não linkadas. Cada uma é prefaciada por um header contendo seu nome, data de criação,
        dono, código de proteção, e tamanho.
    \end{itemize}
    \item Cada SO deve reconhecer pelo menos um tipo de arquivo: seu próprio arquivo executável. Alguns reconhecem mais.
\end{itemize}

\subsubsection{Acesso à Arquivos}
\begin{itemize}
    \item Sistemas operacionais antigos proviam apenas um tipo de acesso a arquivos: \textbf{acesso sequencial}. Nesse tipo
    de sistema, um processo poderia ler todos os bytes ou records de um arquivo em orde, começando do começo, mas não poderia
    pular bytes ou lê-los fora de ordem. Eles podiam ser rebobinados, para poder ser lidos quantas vezes necessários. Esses
    arquivos eram convenientes pois o tipo de armazenamento usado era fita magnética, ao invés de disco.
    \item Com a popularização dos discos para armazenamento de arquivos, tornou-se possível ler os bytes ou records de um 
    arquivo fora de ordem, ou acessá-los por chave ao invés de posição. Esse tipo de acesso se chama \textbf{acesso aleatório},
    e é necessário para muitas aplicações.
\end{itemize}

\subsubsection{Atributos de Arquivos}
\begin{itemize}
    \item Cada arquivo tem um nome e seus dados. Aĺém disso, todso os sistemas operacionais associam outras informações a cada
    arquivo, como por exemplo a data e hora em que o arquivo foi modificado pela última vez e o tamanho do arquivo. Esses
    itens a mais são chamados de \textbf{atributos} ou \textbf{metadados}. A lista de atributos varia consideravelmente
    de sistema para sistema. Algumas possibilidades restão na tabela 1, mas existem outras.
    \begin{table}[!ht]
        \centering
        \caption{Alguns possíveis atributos de arquivos}
        \label{atributos}
        \begin{tabular}{lllll}
        \textbf{Atributo}       & \textbf{Significado}                                     &  &  &  \\
        Proteção                & Quem pode acessar o arquivo e de que forma               &  &  &  \\
        Criador                 & ID da pessoa que criou o arquivo                         &  &  &  \\
        Dono                    & Dono atual                                               &  &  &  \\
        Read-only flag          & 0 para read/write; 1 para read only                      &  &  &  \\
        Hidden flag             & 0 para arquivos normais; 1 p/ arquivos ocultos           &  &  &  \\
        System flag             & 0 p/ arquivos normais; 1 p/ arquivos do sistema          &  &  &  \\
        Archive flag            & 0 p/ possui backup; 1 p/ precisa de backup               &  &  &  \\
        ASCII/Binary flag       & 0 p/ arquivo ASCII; 1 p/ arquivo binário                 &  &  &  \\
        Random Access flag      & 0 p/ acesso sequencial; 1 p/ acesso aleatório            &  &  &  \\
        Temporary flag          & 0 p/ normal; 1 p/ deletar arquivo ao encerrar processo.  &  &  &  \\
        Lock flags              & 0 p/ unlocked; nonzero p/ locked                         &  &  &  \\
        Record length           & Número de bytes num record                               &  &  &  \\
        Key position            & Offset da chave dentro de cada record                    &  &  &  \\
        Key length              & Número de bytes no campo chave                           &  &  &  \\
        Creation time           & Data/hora da criação do arquivo                          &  &  &  \\
        Time of last access     & Data/hora em que o arquivo foi acessado por ultimo       &  &  &  \\
        Time of last change     & Data/hora da última modificação do arquivo               &  &  &  \\
        Current size            & Número de bytes em um arquivo                            &  &  &  \\
        Maximum size            & Tamanho máximo que o arquivo pode atingir                &  &  &  \\
        \end{tabular}
    \end{table}
    \

\end{itemize}

\subsubsection{Operações com arquivos}
Arquivos existem para armazenar informações e permitir que sejam lidas posteriormente. Sistemas 
diferentes provêm operações diferentes para permitir armazenamento e recuperação. As chamadas
de sistemas mais comuns relacionadas a arquivos são:
\begin{enumerate}
    \item Create: O arquivo é criado sem dados. O propósito dessa chamada é anunciar que o arquivo
    está chegando e definir alguns de seus atributos.
    \item Delete: Quando o arquivo não é mais necessário, ele deve ser excluído para liberar
    espaço em disco. Existe sempre uma syscall para esse propósito.
    \item Open: Antes de usar um arquivo, o processo deve abri-lo. O propósito da chamada open
    é permitir que o sistema busque os atributos e uma lista de endereços no disco na memória
    principal para acesso rápido em chamadas posteriores.
    \item Close: Quando todos os processos são encerrados, os atributos e endereços de disco
    não são mais necessários, então o arquivo deve ser fechado para liberar espaço interno na 
    tabela. Muitos sistemas encorajam isso impondo um número máximo de arquivos abertos em
    processos. Um disco é escrito em blocos, e fechar um arquivo força a escrita do último bloco
    do arquvio, mesmo que esse bloco não esteja completamente cheio.
    \item Read: Dados são lidos do arquivo. Geralmente, os bytes vem da posição atual. O chamador
    deve especificar quantos dados são necessários e deve também prover um buffer que irá contê-los.
    \item Write: Dados são escritos novamente no arquivo, geralmente na posição atual. Se a posição atual
    é no final do arquivo, o tamanho do arquivo aumenta. Se a posição atual é no meio do arquivo, dados
    existentes são sobrescritos e perdidos para sempre.
    \item Append: essa chamada é uma forma restrita de escrita. Ela pode adicionar dados apenas ao final
    do arquivo. Ssistemas que provêm um número mínimo de chamadas de sistema raramente tem o append, mas
    muitos sistemas tem várias formas de fazer a mesma coisa, e esses sistemas frequentemente possuem o
    append.
    \item Seek: Para arquivos de acesso aleatorio, um método é necessário para epsecificar de onde buscar
    os dados. Uma forma comum é a chamada seek, que reposiciona o ponteiro do arquivo para um lugar específico
    nesse arquivo. Após essa chamada ser concluída, dados podem ser lidos ou escritos naquela posição.
    \item Get attributes: Processos frequentemente precisam ler atributos dos arquivos para fazer seu trabalho.
    Por exemplo, o programa \textit{make} do UNIX, ao ser chamado, examina as datas/horas de modificação de todso
    os arquivos fonte e objetos e busca fazer o mínimo de compilações necessárias para manter o projeto atualizado.
    \item Set attributes: Alguns atributos podem ser modificados após a criação do arquivo. Essa chamada de sistema
    possibilita isso.
    \item Rename: Frequentemente, um usuário precisa mudar o nome de um arquivo existente. Essa chamada de sistema 
    permite isso. Não é sempre necessário que ela exista, pois o arquivo pode ser copiado para um novo arquivo com
    o novo nome, e depois o arquivo antigo é excluído.
\end{enumerate}

\subsubsection{Exemplo de Programa usando Chamadas de Sistema de Arquivos}
\begin{lstlisting}
/* copyfile.c */
/* File copy program. Error checking and reporting is minimal. */

#include <sys/types.h>
#include <fcntl.h>
#include <stdlib.h>
#include <unistd.h>

int main(int argc, char *argv[]); /* ANSI Prototype */

#define BUF_SIZE 4096             /* use a buffer size of 4096 bytes */
#define OUTPUT_MODE 0700          /* protection bits for output file */

int main(int argc, char *argv[])
{
    int in_fd, out_d, rd_count, wt_count;

    if (argc != 3) exit(1);       /* syntax error if argc is not 3 */


    /* Open the input file and create the output file */

    in_fd = open(argv[1], O_RDONLY); /* open the source file */
    if (in_fd < 0) exit(2);       /*if it cannot be opened, exit */
    out_fd = creat(argv[2], OUTPUT_MODE); /* create the destination file */
    if (out_fd < 0) exit(3);      /* if it cannot be created exit */

    /* Copy loop */
    while(1) {
        rd\_count = read(in_fd, buffer, BUF_SIZE); /* read a block of data */
        if (rd_count <= 0) break; /*if end of file or error, exit loop */
        wt_count = write (out_fd, buffer, rd_count); /*write data*/
        if (wt_count <= 0)exit(4);  /*wt_count <=0 is an error */
    }
    /* Close the files */
    close(in_fd);
    close(out_fd);
    if (rd_count == 0)
        exit(0);    /* no error on last read */
    else
        exit(5);    /* error on last read */
}
\end{lstlisting}
 
 O programa, \textit{copyfile}, pode ser chamado, por exemplo, pela linha de comando
 \begin{lstlisting}
    copyfile abc xyz
 \end{lstlisting}
 para copiar o arquivo abc para xyz. Se o arquivo xyz já existe, será sobrescrito. 
 Caso contrário, será criado. O programa deve ser chamado com exatamente dois argumentos,
 ambos nomes aceitáveis de arquivos. O primeiro é o fonte, o segundo é o destino.

 O primeiro \textit{\#define} define BUF\_SIZE para 4096. Esse valor será usado para determinar
 que a leitura ocorrerá em chunks de 4096 bytes. O segundo \textit{\#define} determina quem
 poderá acessar o arquivo de saída.

 São necessários 3 parâmetros nesse programa: o primeiro é o próprio programa (./copyfile),
 o segundo é o arquivo fonte, e o terceiro, o arquivo destino. Por isso, o programa checa
 se recebeu exatamente 3 argumentos, caso contrário, encerra o programa.

 Os valores de argv seriam, na chamada exemplo:

 \begin{lstlisting}
    argv[0] = "copyfile"
    argv[1] = "abc"
    argv[2] = "xyz"
 \end{lstlisting}
 É através desse array que o programa acessa seus argumentos.

 Cinco variáveis são delcaradas. As variáveis \textit{in\_fd} e \textit{out\_fd}, vão
 armazenar os \textbf{descritores de arquivos}, pequenos inteiros retornados quando 
 um arquivo é aberto. As próximas duas \textit{rd\_count} e \textit{wt\_count}, são as
 contagens de bytes retornadas pelas chamadas de systema read e write, respectivamente.
 O último, \textit{buffer}, é o buffer usado para armazenar os dados lidos e suprir
 os dados a serem escritos.

 Após verificar se o número de argumentos está correto, o programa tenta abrir
 o arquivo fonte e criar o arquivo destino. Se o arquivo fonte for aberto com 
 sucesso, o sistema atribui um pequeno inteiro a \textit{in\_fd}, para identificar
 o arquivo. Chamdas subsequentes devem incluir este inteiro para que o sistema
 saiba qual arquivo ele está buscando. De forma similar, se o arquivo destino
 for criado com sucesso, \textit{out\_fd} recebe um valor para identificar esse
 arquivo. O segundo argumento para \textit{creat} define o modo de proteção. Se 
 a abertura ou a criação do arquivo falharem, o descritor de arquivo correspondente
 recebe -1 e o programa é encerrado com um código de erro.

 Here come dat copy loop. O shit, waddup. Ele inicia tentando ler 4KB de dados
 para o \textit{buffer}, chamando a rotina de biblioteca \textit{read}, que invoca
 a chamada de sistema read. O primeiro parâmetro identifica o arquivo, o segundo passa
 o buffer, e o terceiro diz quantos bytes devem ser lidos. O valor atribuido a
 \textit{rd\_count} diz o número de bytes que realmente foram lidos. Normalmente, isso
 será 4096, exceto se menos bytes estiverem restando no arquivo. Quando o fim do arquivo for
 atingido, isso será zero. Se \textit{rd\_count} for zero ou negativo, a cópia não pode continuar,
 então a chamada \textit{break} é executada para terminar o loop.

 A chamada a \textit{write} salva o buffer no arquivo destino. O primeiro parâmetro identifica 
 o arquivo, o segundo passa o buffer, e o terceiro diz quantos bytes serão escritos, de forma
 análoga ao \textit{read}. Note que a contagem de bytes é o número de bytes realmente lidos, e
 não \textit{BUF\_SIZE}. Isso é importante porque a última leitura não retornará 4096 à não ser
 que o tamanho do arquivo seja, por ocasião, múltiplo de 4KB.

 Quando o arquivo inteiro for processado, a primeira chamada após o final do arquivo retornará
 0 ao \textit{rd\_count}, o que fará o arquivo sair do loop. Nesse ponto, os dois arquivos são fechados
 e o programa termina com um status indicando um término normal do arquivo. 

 \subsection{Diretórios}

 Para manter os arquivos organizados, os sistemas de arquivos normalmente usam \textbf{diretórios}
 ou \textbf{pastas}, que são arquivos também. 

 \subsubsection{Sistemas de diretório single-level}
 A forma mais simples de um sistema de diretórios é ter um único diretório contendo todos os
 arquivos, normalmente chamado de \textbf{root directory}. Esse sistema era comum em PCs antigos.

 \subsubsection{Sistemas de diretório hierárquivos}
 Para usuários modernos que trabalham com milhares de arquivos, encontrar qualquer coisa seria
 impossível se todos os arquivos estivessem em um único diretório. Por isso, uma forma é necessária
 para agrupar todos os arquivos relacionados. O que se precisa é de hirarquia (ou seja, uma árvore
 de diretórios). Com essa aproximação, podem haver tantos diretorios quanto forem necessários para
 agrupar os arquivos em formas naturais. Quase todos os sistemas de arquivos modernos são organizados
 desta maneira.

 \subsubsection{Nomes de caminho}
 Quando o sistema de arquivos é organizado como uma árvore de diretórios, alguma forma é necessária
 para especirficar os nomes dos arquivos. Dois métodos diferentes são comumente usados. No primeiro
 método, cada arquivo recebe um \textbf{nome de caminho absoluto} consistindo do caminho desde o 
 diretório raiz até o arquivo. Caminhos absolutos sempre começam no diretório root e são únicos.
 No UNIX, os componentes do arquivo são separados por um /. 

 O outro tipo de nome de caminho é o \textbf{nome de caminho relativo}. Esse é usado em conjunto
 com o conceito de \textbf{diretório atual}. Um usuário pode designar um diretório como sendo
 o diretório atual, nesse caso, todos os caminhos que não comecem a partir do diretório raiz são
 relativos ao diretório atual. Por exemplo, se o diretório atual for \textit{/usr/ast}, então o 
 arquivo cujo caminho absoluto é \textit{/usr/ast/mailbox} pode ser referenciado simplesmente como
 \textit{mailbox}. Em outras palavras, o comando UNIX 
 \begin{lstlisting}
    cp /usr/ast/mailbox /usr/ast/mailbox.bak
 \end{lstlisting}
 e o comando 
\begin{lstlisting}
    cp mailbox mailbox.bak
\end{lstlisting}
fazem exatamente a mesma coisa, se o diretório atual for \textit{/usr/ast}. Em alguns casos,
usar o diretório atual pode ser mais conveniente. Em outros casos, pode ser necessário saber
o endereço absoluto de um arquivo a ser acessado, independente do diretório atual.

Cada processo tem seu próprio diretório atual, então quando ele muda seu diretório atual e depois
termina, nenhum outro processo é afetado e nenhum traço da mudança é deixado para trás no sistema
de arquivos. Dessa forma, é sempre perfeitamente seguro para um processo mudar seu diretório atual
sempre que achar conveniente. 

Entretanto, se uma \textit{rotina de biblioteca} muda o diretório atual e não retorna ao original
antes de terminar, o resto do programa pode não funcionar, já que o endereço atual pode não estar
válido. Por isso, rotinas de biblioteca raramente mudam o diretório atual, e quando elas o fazem,
sempre mudam de volta antes de retornar.

Quase todos os sistemas operacionais que suportam uma hirarquia de diretórios possuem duas entradas
especiais em todos os diretórios, "." e "..". "." se refere ao diretório atual, e ".." se refere ao
diretório pai (exceto no diretório raiz, onde se refere a si mesmo).

\subsubsection{Operações com diretórios}

As chamadas de sistema permitidas para gerenciar diretórios exibem mais variação de sistema para
sistema do que as chamadas de sistema para arquivos. Exemplo retirado do UNIX.
\begin{enumerate}
    \item Create: Um diretório é criado. Ele está vazio exceto por "." e "..", que são automaticamente
    colocados lá pelo sistema (ou em alguns casos, pelo programa \textit{mkdir}).
    \item Delete: Um diretório é excluído. Apenas um diretório vazio pode ser excluído. Um diretório contendo
    apenas "." e ".." é considerado vazio, já que esses dois não podem ser excluídos.
    \item Opendir: Diretórios podem ser lidos. Por exemplo, para listar todos os arquivos em um diretório, um 
    programa de listagem abre o diretório para ler os nomes de todos os arquivos que ele contém. Antes que
    um diretório possa ser lido, ele deve ser aberto, de forma análoga a abertura e leitura de um arquivo.
    \item Closedir: Quando um diretório foi lido, ele deve ser fechado para liberar espaço interno na tabela.
    \item Readdir: Essa chamada retorna a próxima entrada em um diretório aberto. Anteriormente, era possível ler
    diretórios usando a system call read usual, mas essa chamada apresentava a desvantagem de forçar o programador
    a lidar com a estrutura interna dos diretórios. Readdir sempre retorna uma entrada em um formato padrão, não
    importanto qual das possíveis  estruturas de diretório esteja sendo usada.
    \item Rename: Diretórios podem ser renomeados da mesma forma que arquivos.
    \item Link: Uma técnica que permite que um arquivo apareça em mais de um diretório. Essa chamada de sistema
    especifica um arquivo existente e um nome de caminho, e cria um link do arquivo existente ao nome especificado
    eplo caminho. Dessa forma, o mesmo arquivo pode aparecer em muitos diretórios. Um link desse arquivo, que incrementa
    o contador no inode do arquivo (para controlar quantas entradas de diretório contém o arquivo) é chamado de 
    \textbf{hard link}.
    \item Unlink: Uma entrada de diretório é removida. Se o arquivo a ser removido está somente presente em um diretório,
    ele é removido do sistema de arquivos. Se está presente em múltiplos diretórios, apenas o caminho especificaod é removido,
    e os outros permanecem. No UNIX, essa é a chamada de sistema usada para excluir arquivos.
\end{enumerate}

Uma variante na idéia de linkar arquivos é o \textbf{link simbólico}. Ao invés de fazer com que dois nomes apontem para a 
mesma estrutura intenra de dados que representa um arquivo, um link simbólico cria um nome que aponta para um
pequeno arquivo que nomeia outro arquivo. Quando o primeiro arquivo é usado, por exemplo, aberto,
o  sistema de arquivos segue o caminho e encontra o nome no final. Então, ele começa o processo de busca
usando o novo nome. Links simbólicos possuem a vantagem de poder cruzar barreiras entre discos e até
mesmo apontar para arquivos em computadores remotos. Porém, sua implementação é menos eficiente
que a de hard links.

\subsection{Implementação de sistemas de arquivos}
Saindo do ponto de vista de usuário, estuda-se agora a visão de um implementador. A preocupação
do usuário se refere nos nomes de arquivos, operações possíveis sobre eles, estrutura dos 
diretórios, etc. O implementador, por outro lado, precisa saber como os arquivos e diretórios são
salvos, como o espaço em disco é armazenado, e como fazer tudo funcionar de forma eficiente e
confiável.

\subsubsection{Layout do Sistema de Arquivos}
\item Sistemas de arquivos são armazenados em discos. A maioria dos discos pode ser divididos
em uma ou mais partições, com sistemas de arquivos independentes em cada partição. O Setor 0
do disco é chamado de \textbf{MBR (Master Boot Record)} e é usado para iniciar o computador.
Ao final do MBR está a tabela de partições. Essa tabela contém os endereços e finais de cada
partiçãso. Uma das partições na tabela é marcada como ativa. Quando o computador é bootado,
a BIOS lê e executa a MBR. A primeira coisa que a MBR faz é procurar a partição ativa, ler
seu primeiro bloco (que é chamado de \textbf{boot block}), e executá-lo. O programa no 
boot block carrega o SO contido naquela partição. Para fins de uniformidade, cada partição
começa com um boot block, mesmo que ela não contenha um sistema operacional bootavel. Ademais,
ela pode conter um no futuro.

Além de iniciar com um boot block, o layout de uma partição de disco varia muito entre os
sistemas de arquivos. É comum que o sistema de arquivos contenha alguns dos seguintes itens:

\begin{enumerate}
    \item \textbf{Superblock}: Contém todos os parâmetros chave sobre o sistema de arquivos
    e é lido para a memória quando o computador é bootado ou o sistema de arquivos é tocado
    pela primeira vez. Informações típicas no superblock incluem um número mágico para 
    identificar o tipo do sistema de arquivos, o número de blocos no sistema de 
    arquivos, e outras informações administrativas chave.
    \item Em seguida, podem existir informações a respeito dos blocos livres no sistema
    de arquivos, por exemplo, na forma de um bitmap ou lista de ponteiros. 
    \item Após isso, podem estar os i-nodes, um array de estruturas de dados, uma por arquivo,
    que contém todas as informações sobre aquele arquivo.
    \item Depois, pode estar o diretório raiz, que contém o topo da árvore do sistema de arquivos.
    \item Finalmente, o restante da partição contém todos os outros diretórios e arquivos.
\end{enumerate}

\subsubsection{Implementação de arquivos}

Provavelmente o problema mais importante em implementar armazenamento de arquivos é manter
controle sobre quais blocos de disco vão para cada arquivo. Vários métodos são usados em 
diferentes SOs.

\begin{itemize}
    \item \textbf{Alocação Contígua}
    A forma de alocação mais simples é armazenar cada arquivo como uma série
    contígua de blocos de disco. Portanto, em um disco com blocos de 1KB, um arquivo de 
    50KB teria 50 blocos consecutivos alocados para seu uso. Com blocos de 2kb, teria 
    25 blocos consecutivos.

\end{itemize}

\end{document}
