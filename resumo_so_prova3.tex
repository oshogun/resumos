\documentclass[10pt]{article}
\usepackage[utf8]{inputenc}
\usepackage[T1]{fontenc}
\usepackage[brazilian]{babel}
\usepackage{comment} 
\usepackage{geometry}
\geometry{
    a4paper,
    total={170mm,257mm},
    left=20mm,
    top=20mm,
 }
\setlength{\parindent}{4em}
\setlength{\parskip}{1em}
\renewcommand{\baselinestretch}{1.2}

\title{Resumo Prova 3 SO}
\author{Guilherme Christopher Michaelsen Cardoso}
\date{\today}

\begin{document}
\maketitle

\section{Sistemas de Arquivos}
\begin{itemize}
    \item Programas de computador precisam armazenar e recuperar informação.
    \begin{itemize}
        \item Quando um processo está rodando, ele pode armazenar uma quantidade
        limitada de informação dentro do seu próprio espaço de endereçamento.
        \item Porém, a capacidade desse armazenamento está limitada ao tamanho
        do espaço de endereçamento virtual. 
        \item Para alguns programas, esse tamanho é adequado; para outros, é muito
        pequeno.
        \item Outro problema de se utilizar o espaço de enderaçmento virtual é
        que ao encerrar o processo, a informação é perdida. Para muitas aplicações,
        a informação deve ser retida por semanas, meses, ou mesmo para sempre. 
        Não se aceita que essa informação desapareça quando um processo termina ou
        sofre um \textit{crash}.
        \item O terceiro problema é que frequentimente é necessário que múltiplos
        processos acessem partes da informação ao mesmo tempo. Por isso, a informação
        deve ser independente dos processos.
    \end{itemize}
    \item Assim, surgem 3 requisitos essenciais para armazenamento de informação à 
    longo prazo:
    \begin{enumerate}
        \item Deve ser possivel armazenar uma quantidade muito grande de informação
        \item A informação deve sobreviver ao término do processo que a está utilizando
        \item Múltiplos processos devem ser capazes de acessar a informação de uma vez.
    \end{enumerate}
    \item Dispositivos comuns de armazenamento não-volátil incluem:
    \begin{enumerate}
        \item Discos magnéticos
        \item SSDs (não possuem partes móveis que podem quebrar, oferecem acesso rápido)
        \item Fitas e discos ópticos (usados tipicamente para backup devido a sua baixa
        performance).
    \end{enumerate}
    \item Pode-se pensar no disco como sendo uma sequência linear de blocos de tamanho
    fixo, capazes de suportar duas operações:
    \begin{enumerate}
        \item Ler o bloco \textit{k}
        \item Escrever no bloco \textit{k}
    \end{enumerate}
    \begin{itemize}
        \item Na verdade existem mais operações, mas essas duas poderiam, em princípio,
        resolver o problema do armazenamento à longo prazo.
    \end{itemize}
\end{itemize}

\end{document}
