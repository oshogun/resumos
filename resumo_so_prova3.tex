\documentclass[10pt]{article}
\usepackage[utf8]{inputenc}
\usepackage[T1]{fontenc}
\usepackage[brazilian]{babel}
\usepackage{comment} 
\usepackage{geometry}
\geometry{
    a4paper,
    total={170mm,257mm},
    left=20mm,
    top=20mm,
 }
\setlength{\parindent}{4em}
\setlength{\parskip}{1em}
\renewcommand{\baselinestretch}{1.2}

\title{Resumo Prova 3 SO}
\author{Guilherme Christopher Michaelsen Cardoso}
\date{\today}

\begin{document}
\maketitle

\section{Sistemas de Arquivos}
\begin{itemize}
    \item Programas de computador precisam armazenar e recuperar informação.
    \begin{itemize}
        \item Quando um processo está rodando, ele pode armazenar uma quantidade
        limitada de informação dentro do seu próprio espaço de endereçamento.
        \item Porém, a capacidade desse armazenamento está limitada ao tamanho
        do espaço de endereçamento virtual. 
        \item Para alguns programas, esse tamanho é adequado; para outros, é muito
        pequeno.
        \item Outro problema de se utilizar o espaço de enderaçmento virtual é
        que ao encerrar o processo, a informação é perdida. Para muitas aplicações,
        a informação deve ser retida por semanas, meses, ou mesmo para sempre. 
        Não se aceita que essa informação desapareça quando um processo termina ou
        sofre um \textit{crash}.
        \item O terceiro problema é que frequentimente é necessário que múltiplos
        processos acessem partes da informação ao mesmo tempo. Por isso, a informação
        deve ser independente dos processos.
    \end{itemize}
    \item Assim, surgem 3 requisitos essenciais para armazenamento de informação à 
    longo prazo:
    \begin{enumerate}
        \item Deve ser possivel armazenar uma quantidade muito grande de informação
        \item A informação deve sobreviver ao término do processo que a está utilizando
        \item Múltiplos processos devem ser capazes de acessar a informação de uma vez.
    \end{enumerate}
    \item Dispositivos comuns de armazenamento não-volátil incluem:
    \begin{enumerate}
        \item Discos magnéticos
        \item SSDs (não possuem partes móveis que podem quebrar, oferecem acesso rápido)
        \item Fitas e discos ópticos (usados tipicamente para backup devido a sua baixa
        performance).
    \end{enumerate}
    \item Pode-se pensar no disco como sendo uma sequência linear de blocos de tamanho
    fixo, capazes de suportar duas operações:
    \begin{enumerate}
        \item Ler o bloco \textit{k}
        \item Escrever no bloco \textit{k}
    \end{enumerate}
    \begin{itemize}
        \item Na verdade existem mais operações, mas essas duas poderiam, em princípio,
        resolver o problema do armazenamento à longo prazo.
        \item Na realidade, essas são operações muito inconvenientes, especialmente
        em sistemas grandes usado por muitas aplicações e, possivelmente, por muitos
        usuários (por exemplo, em um servidor). Alguns problemas que surgem são:
        \begin{enumerate}
            \item Como encontrar informação?
            \item Como garantir que um usuário não leia os dados de outro?
            \item Como saber quais blocos estão livres?
        \end{enumerate}
        e existem muito mais.
    \end{itemize}
    \item Da mesma forma que o SO abstraiu o conceito do processador para criar o 
    conceito do processo, e abstraiu o conceito da memória física para oferecer
    a processos espaços de endereçamento virtuais, os problemas relacionados a disco
    podem ser resolvidos com uma nova abstração: \textbf{o arquivo}. 
    \begin{itemize}
        \item Esses três conceitos (processos, espaços de endereçamento e arquivos)
        são os mais importantes em Sistemas Operacionais.
    \end{itemize}
    \item \textbf{Arquivos} são unidades lógicas de informação criadas pelos processos.
    Um disco normalmente contém milhares ou até mesmo milhões deles, cada um 
    independente dos outros. Assim como os espaços de endereçamento modelam a RAM,
    arquivos modelam o disco.
    \item Processos podem ler arquivos existentes e criar novos arquivos de acordo com 
    a necessidade. Informação armazenada em arquivos deve ser persistente, isto é, não
    pode ser afetatda pela criação e término dos processos. Um arquivo deve desaparecer
    apenas quando seu usuário explicitamente o remove. Apesar de operações de leitura
    e escrita de arquivos serem as mais comuns, existem muitas outras.
    \item Arquivos são gerenciados pelo sistema operacional. A forma em que eles são
    estruturados, nomeados, acessados, usados, protegidos, implementados e gerenciados
    são tópicos importantes em design de sistemas operacionais. A parte do SO que lida
    com arquivos é chamada de \textbf{sistema de arquivos}.
\end{itemize}
\subsection{Arquivos}
\abstract{Nessa subseção, observa-se os arquivos do ponto de vista do usuário,
isto é, como eles serão usados e quais propriedades eles terão.}
\subsubsection{Nomeação de Arquivos}
\par Lorem ipsum dolor sit amet presuntum queijum abaj duwang shogun kakarotto
sugoi desu ne nani rabusekku

\end{document}
