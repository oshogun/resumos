\documentclass[10pt]{article}
\usepackage[utf8]{inputenc}
\usepackage[T1]{fontenc}
\usepackage[brazilian]{babel}
\usepackage{comment} 
\usepackage{geometry}
\geometry{
    a4paper,
    total={170mm,257mm},
    left=20mm,
    top=20mm,
 }
\setlength{\parindent}{4em}
\setlength{\parskip}{1em}
\renewcommand{\baselinestretch}{1.2}

\title{Resumo Prova 3 SO}
\author{Guilherme Christopher Michaelsen Cardoso}
\date{\today}

\begin{document}
\maketitle

\section{Sistemas de Arquivos}
\begin{itemize}
    \item Programas de computador precisam armazenar e recuperar informação.
    \begin{itemize}
        \item Quando um processo está rodando, ele pode armazenar uma quantidade
        limitada de informação dentro do seu próprio espaço de endereçamento.
        \item Porém, a capacidade desse armazenamento está limitada ao tamanho
        do espaço de endereçamento virtual. 
        \item Para alguns programas, esse tamanho é adequado; para outros, é muito
        pequeno.
        \item Outro problema de se utilizar o espaço de enderaçmento virtual é
        que ao encerrar o processo, a informação é perdida. Para muitas aplicações,
        a informação deve ser retida por semanas, meses, ou mesmo para sempre. 
        Não se aceita que essa informação desapareça quando um processo termina ou
        sofre um \textit{crash}.
        \item O terceiro problema é que frequentimente é necessário que múltiplos
        processos acessem partes da informação ao mesmo tempo. Por isso, a informação
        deve ser independente dos processos.
    \end{itemize}
    \item Assim, surgem 3 requisitos essenciais para armazenamento de informação à 
    longo prazo:
    \begin{enumerate}
        \item Deve ser possivel armazenar uma quantidade muito grande de informação
        \item A informação deve sobreviver ao término do processo que a está utilizando
        \item Múltiplos processos devem ser capazes de acessar a informação de uma vez.
    \end{enumerate}
    \item Dispositivos comuns de armazenamento não-volátil incluem:
    \begin{enumerate}
        \item Discos magnéticos
        \item SSDs (não possuem partes móveis que podem quebrar, oferecem acesso rápido)
        \item Fitas e discos ópticos (usados tipicamente para backup devido a sua baixa
        performance).
    \end{enumerate}
    \item Pode-se pensar no disco como sendo uma sequência linear de blocos de tamanho
    fixo, capazes de suportar duas operações:
    \begin{enumerate}
        \item Ler o bloco \textit{k}
        \item Escrever no bloco \textit{k}
    \end{enumerate}
    \begin{itemize}
        \item Na verdade existem mais operações, mas essas duas poderiam, em princípio,
        resolver o problema do armazenamento à longo prazo.
        \item Na realidade, essas são operações muito inconvenientes, especialmente
        em sistemas grandes usado por muitas aplicações e, possivelmente, por muitos
        usuários (por exemplo, em um servidor). Alguns problemas que surgem são:
        \begin{enumerate}
            \item Como encontrar informação?
            \item Como garantir que um usuário não leia os dados de outro?
            \item Como saber quais blocos estão livres?
        \end{enumerate}
        e existem muito mais.
    \end{itemize}
    \item Da mesma forma que o SO abstraiu o conceito do processador para criar o 
    conceito do processo, e abstraiu o conceito da memória física para oferecer
    a processos espaços de endereçamento virtuais, os problemas relacionados a disco
    podem ser resolvidos com uma nova abstração: \textbf{o arquivo}. 
    \begin{itemize}
        \item Esses três conceitos (processos, espaços de endereçamento e arquivos)
        são os mais importantes em Sistemas Operacionais.
    \end{itemize}
    \item \textbf{Arquivos} são unidades lógicas de informação criadas pelos processos.
    Um disco normalmente contém milhares ou até mesmo milhões deles, cada um 
    independente dos outros. Assim como os espaços de endereçamento modelam a RAM,
    arquivos modelam o disco.
    \item Processos podem ler arquivos existentes e criar novos arquivos de acordo com 
    a necessidade. Informação armazenada em arquivos deve ser persistente, isto é, não
    pode ser afetatda pela criação e término dos processos. Um arquivo deve desaparecer
    apenas quando seu usuário explicitamente o remove. Apesar de operações de leitura
    e escrita de arquivos serem as mais comuns, existem muitas outras.
    \item Arquivos são gerenciados pelo sistema operacional. A forma em que eles são
    estruturados, nomeados, acessados, usados, protegidos, implementados e gerenciados
    são tópicos importantes em design de sistemas operacionais. A parte do SO que lida
    com arquivos é chamada de \textbf{sistema de arquivos}.
\end{itemize}
\subsection{Arquivos}
\abstract{Nessa subseção, observa-se os arquivos do ponto de vista do usuário,
isto é, como eles serão usados e quais propriedades eles terão.}
\subsubsection{Nomeação de Arquivos}
\begin{itemize}
    \item Um arquivo é um mecanismo de abstração, que provê uma forma de armazenar
    informação no disco e recuperá-la depois. 
    \begin{itemize}
        \item Isso deve ocorrer de forma a esconder do usuário os detalhes sobre como
        e onde essa informação é armazenada, e como os discos funcionam.
    \end{itemize}
    \item Quando um processo cria um arquivo, ele dá a esse arquivo um nome.
    \begin{itemize}
        \item Quando o processo termina, esse arquivo continua a existir e pode ser acessado
        por outros processos usando seu nome.
    \end{itemize}
    \item As regras para nomear arquivos variam de sistema para sistema.
    \begin{itemize}
        \item Todos os SOs atuais permitem strings de 1 a 8 letras como nome de arquivo.
        \item Frequentemente digitos numéricos e caracteres especiais também são permitidos.
        \item Muitos sistemas de arquivos suportam nomes longos de até 255 caracteres.
        \item Alguns sistemas de arquivos realizam distinção entre letras maiúsculas
        e minúsculas (UNIX, por exemplo), e outras não (MS-DOS, por exemplo).
        \begin{itemize}
            \item Um adendo sobre sistemas de arquivos: Tanto o Windows 95 quanto o 
            Windows 98 usam o sistema de arquivos do DOS, chamado de \textbf{FAT-16},
            herdando muitas de suas propriedades (como a construção de nomes de arquivos).
            Windows 98 introduziu algumas extensões ao FAT-16, criando o \textbf{FAT-32},
            mas esses dois são similares. As versões posteriores do Windows suportam os 
            sistemas de arquivos FAT, que já são obsoletos, porém, esses sistemas já 
            utilizam um sistema mais avançado (\textbf{NTFS}), que possui propriedades
            diferentes (como nomes de arquivo em Unicode). No Windows 8 ainda existe um
            segundo sistema de arquivos, chamado \textbf{ReFS}, mas esse é usado apenas
            na versão para servidores. Nesse capítulo, ao mencionar aos sistemas de arquivos
            do MS-DOS ou FAT, está se referenciando o uso dos sistemas FAT-16 e FAT-32 no 
            Windows. Ainda existe um outro sistema baseado em FAT, conhecido como \textbf{exFAT},
            que é uma extensão do FAT-32 otimizada para pendrives e sistemas de arquivos grandes.
        \end{itemize}
        \item Muitos sistemas operacionais suportam nomes de arquivos dividodos em duas partes,
        separadas por um ponto (ex.: "prog.c"). A parte depois do ponto é chamada de 
        \textbf{extensão de arquivo} e geralmente indica algo sobre o arquivo. No MS-DOS, por
        exemplo, nomes de arquivos contém de 1 a 8 caracters, mais uma extensão adicional de
        1 a 3 caracteres. No UNIX, o tamanho da extensão é definido pelo usuário, e um arquivo
        pode ter até mesmo mais de uma extensão (exemplo homepage.html.zip).
        \begin{itemize}
            \item No caso do UNIX e alguns outros sistemas, as extensões de arquivo são apenas
            convenções e não são obrigatórias pelo sistema operacional. Um arquivo chamado
            file.txt pode ser algum tipo de arquivo de texto, mas o nome serve mais
            para informar o usuário do que para prover qualquer informação para o computador.
            Por outro lado, um compilador C pode exigir que os arquivos de entrada estejam 
            com extensão .c, mas o SO não se importa.
            \item Essas convenções são especialmente úteis quando o mesmo programa pode 
            lidar com diversos tipos de arquivos.
            \item Por outro lado, o Windows utiliza as extensões para identificar quais programas
            são "donos" de cada extensão. Quando o usuário chama um arquivo, o SO invoca o programa
            associado à sua extensão.
        \end{itemize}
    \end{itemize}
\end{itemize}
\subsubsection{Estrutura de Arquivos}
Arquivos podem ser estruturados de várias formas. Três possibilidades são comuns:
\begin{enumerate}
    \item Uma sequência não-estruturada de bytes. O SO não se preocupa com o que está no arquivo.
    Tudo que ele vê são os bytes. Qualquer significado deve ser imposto pelos programas à nível de
    usuário. Tanto o UNIX quanto o Windows usam essa aproximação. 

    Fazer com que o SO considere os arquivos como nada mais do que sequências de bytes provê a 
    maior flexibiliade. Programas de usuário podem colocar qualquer coisa em seus arquivos e nomeá-los
    de qualquer forma que achar conveniente. O SO não ajuda, mas também não atrapalha. Para usuários
    que querem fazer coisas incomuns, isso é muito importante. Todas as versões do UNIX (incluindo
    Linux e OSX) e o Windows usam esse modelo de arquivos.

    \item Uma sequência de records de tamanho fixo, cada um com uma estrutura interna. I idéia central
    de um arquivo ser uma sequencia de records é a idéia de que a operação de leitura retorna um record
    e a operação de escrita substitui ou acrescenta um record. Essa técnica era usada quando cartões perfurados
    de 80 colunas eram comuns, e os SOs usavam arquivos que consistiam de records de 80 caracteres.

    \item Uma árvore de records, não necessariamente todos do mesmo tamanho, cada um contendo um campo 
    \textbf{chave} em uma posição fixa do record. Essa árvore é ordenada no campo chave, para permitir
    a busca rápida de uma determinada chave. Esse tipo de arquivo é usado em computadores mainframe
    para processamento de dados comerciais.
\end{enumerate}

\subsubsection{Tipos de Arquivos} 

\begin{itemize}
    \item Vários SOs suportam vários tipos de arquivos.
    \item UNIX e Windows, por exemplo, possuem arquivos regulares e diretórios. O UNIX também possui 
    arquivos especiais de caracteres e blocos. 
    \begin{itemize}
        \item \textbf{Arquivos regulares} contém informação de usuário. 
        \item \textbf{Diretórios} são arquivos de sistema que mantém a estrutura do sistema de arquivos.
        \item \textbf{Arquivos especiais de caracteres} são relacionados à entrada/saída e são usados
        para modelar dispositivos seriais de I/O como terminais, impressoras e redes.
        \item \textbf{Arquivos especiais de bloco} são usados para modelar discos.
    \end{itemize}
    \item Arquivos regulares geralmente são ASCII ou binários. Arquivos ASCII contém linhas de texto.
    Em alguns sistemas, cada linha é terminada por um caractere de carriage return. Em outros, o caractere
    de line feed é usado. Em alguns sistemas (e.g., Windows), ambos são usados. Linhas não precisam ser todas
    do mesmo comprimento.
    \item A grande vantagem de arquivos ASCII é que eles podem ser mostrados e impressos como são, e podem ser
    editados com qualquer editor de texto. Além disso, se grandes números de programas usam arquvios ASCII 
    para entrada e saída, é facil conectar a saída de um programa na entrada de outro (como nos pipelines
    de shell).
    \item Arquivos binários são aqueles que não são ASCII. Tentar imprimir seu conteúdo na tela geraria
    um monte de lixo aleatório. Geralmente eles tem alguma estrutura interna conhecida apenas pelos programas
    que os utilizam. 
    \begin{itemize}
        \item Um exemplo de arquivo binário seria um executável do UNIX. Apesar de tecnicamente ser apenas
        uma sequência de bytes, o SO vai executar um arquivo apenas se el tiver um formato apropriado, com
        cinco sessões: header, text, data, relocation bits e symbol tabe. O header começa com um 
        \textbf{magic number} que identifica o arquivo como um executável (para impedir a execução acidental
        de um arquivo que não esteja nesse formato), seguido pelos tamanhos das várias partes do arquivo, o
        endereço no qual a execução começa, e alguns bits de flag. Em seguida estão os segmentos de texto e dados
        do programa em si. Eles são carregados em memória e relocados usando os bits de relocação. A tabela de símbolos
        é usada apenas para debug.
        \item Outro exemplo seria um arquivo compactado, também do UNIX. Ele consiste de uma coleção de rotinas de 
        bibliotecas compiladas mas não linkadas. Cada uma é prefaciada por um header contendo seu nome, data de criação,
        dono, código de proteção, e tamanho.
    \end{itemize}
    \item Cada SO deve reconhecer pelo menos um tipo de arquivo: seu próprio arquivo executável. Alguns reconhecem mais.
\end{itemize}

\subsubsection{Acesso à Arquivos}
\begin{itemize}
    \item Sistemas operacionais antigos proviam apenas um tipo de acesso a arquivos: \textbf{acesso sequencial}. Nesse tipo
    de sistema, um processo poderia ler todos os bytes ou records de um arquivo em orde, começando do começo, mas não poderia
    pular bytes ou lê-los fora de ordem. Eles podiam ser rebobinados, para poder ser lidos quantas vezes necessários. Esses
    arquivos eram convenientes pois o tipo de armazenamento usado era fita magnética, ao invés de disco.
    \item Com a popularização dos discos para armazenamento de arquivos, tornou-se possível ler os bytes ou records de um 
    arquivo fora de ordem, ou acessá-los por chave ao invés de posição. Esse tipo de acesso se chama \textbf{acesso aleatório},
    e é necessário para muitas aplicações.
\end{ìtemize}

\subsubsection{Atributos de Arquivos}
\begin{itemize}
    \item Cada arquivo tem um nome e seus dados. Aĺém disso, todso os sistemas operacionais associam outras informações a cada
    arquivo, como por exemplo a data e hora em que o arquivo foi modificado pela última vez e o tamanho do arquivo. Esses
    itens a mais são chamados de \textbf{atributos} ou \textbf{metadados}. A lista de atributos varia consideravelmente
    de sistema para sistema. Algumas possibilidades estão na tabela abaixo, mas existem outras.
    \begin{table}[]
        \centering
        \caption{My caption}
        \label{my-label}
        \begin{tabular}{lllll}
        \textbf{Atributo} & \textbf{Significado}                       &  &  &  \\
        Proteção          & Quem pode acessar o arquivo e de que forma &  &  &  \\
        Criador           & ID da pessoa que criou o arquivo           &  &  &  \\
        Dono              & Dono atual                                 &  &  & 
        \end{tabular}
    \end{table}
    
\end{itemize}
\end{document}
